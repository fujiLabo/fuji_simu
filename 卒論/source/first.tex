\section{はじめに}
\label{tag:first}
 近年、多くの企業や教育機関においてLMS(Learning Management System)を用いてeラーニングが行われている。しかし、LMSが行うのは学習の管理であり、教材や資料の配布、簡単なテストや課題の実施に、それに対する評価を行うのが主な仕事である。より高度な学習をLMS上で行うには、学習したい内容に合わせた学習支援ツールをLMSに導入しなければならない。この学習支援ツールは特定のLMS上での動作を想定して設計されており、同一のLMS上でしか動作できず、また、LMS上で動作するためには逐一学習支援ツールをインストールし、プラグインとして動作するための細かな設定を行わなければならない。\\
 また、インターネットの普及が進むにつれ、情報技術者にとってネットワーク技術への理解は必要不可欠なものであると同時に、座学などを用いて知識としてネットワーク技術を学習しても、実際のネットワークと学習したネットワーク技術の知識が繋がりづらい分野である。そこで、実際にネットワークを構築し、機器情報などを追加することで、自らの手で正しいネットワークを形成する演習を行うことが実際のネットワークとそれに付随する知識を深めるのに効果的だと考えられる。しかし、教育機関や学習者である個人が、ネットワーク構築の演習に必要な機器をすべて揃え、それらを用いてネットワークの構築を行うのは、あまり現実的ではない。\\
 これらの問題を解決するため、本研究では、LTI(Learning Tools Interoperability)に準拠した学習支援ツールとして、ネットワーク自己学習機能を保持したWebアプリケーションの実装を提案する。LTIに準拠した学習支援ツールであれば、LTIに準拠したLMSから呼び出すことができる。これによって逐一インストールする必要がなく、学習支援ツールは独立したWebアプリケーションとして機能しているのでLTIに準拠したLMSならば、様々なLMSから呼び出すことが可能である。本研究では異なる仮想マシン上にLTIに準拠したLMSとしてCanvasとMoodleをそれぞれ導入し、LTIに準拠した学習支援ツールとしてネットワーク自己学習機能を持ったネットワークシミュレータを導入した。異なるLMSであるCanvasとMoodleからLTIに準拠した学習支援ツールであるネットワークシミュレータの機能を同じように使用し、ネットワークシミュレータでの動作に応じた得点をLMS側に反映することでLTIに準拠したネットワーク自己学習機能の実装とした。
