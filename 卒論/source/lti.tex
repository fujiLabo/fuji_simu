\section{LTI}
\label{tag:LTI}


ここではLTIについて述べさせてもらう。
LTI(
Learning Tools IterOperability)とは、IMS Global Learning Consortium(以下、IMTと呼ぶ)が、異なるプラットフォーム間(異なるLMS上)における学習支援ツールの相互運用を可能とする技術に関する企画を策定し、標準化した規格のことである。LTIに準拠することの具体的なイメージとして、次のようなケースを想定することができる。先代の研究によりできたNSFをツール・プロバイダとし、異なるLMSから利用するケース。
これにより、LTIに準拠することでMoodle、やCanvasなどの異なるLMS間でNSFとの連携を取ることができた。
本研究ではMoodle、Canvasでの起動を行った。

LTIに置ける用語
Tool Provider(ツール・プロバイダ)
Tool Provider(ツール・プロバイダ)とは、外部ツールや外部コンテンツのことであり、本研究ではNSFがツールプロバイダとなる。

Tool Consumer(ツール・コンシューマ)
Tool Consumer(ツール・コンシューマ)とは、ツール・プロバイダから提供されたツールを使用するLMSのことである。ツール・コンシューマは例として、Canvas,Moodle,Sakai,blackbordなどがある。本研究ではMoodle、Canvasを使用した。

LTIに置ける利点
LTI化することにより、様々なLMSからログインすることなくLMSの学習支援ツールとして利用することができる。また、学習支援ツールを、異なるLMSに合わせた設計で作らずに済むことも利点としてあげられる。これによりツール製作者はツールの再利用および、ツールの共有を可能とすることができる。

LTIの利用方法

LMS上でLTIに準拠したツール・プロバイダを使用するには、各LMS上で外部ツールの設定を変更する必要がある。
例として、moodleでは
