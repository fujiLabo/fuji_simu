\section*{\center 概 要}
\addcontentsline{toc}{section}{概要}

インターネットの普及に伴い、情報技術者にとってTCP/IPを中心としたネットワーク技術の理解は必要不可欠である。ネットワークの構築演習として実機を使用した演習があるが、学習者一人ひとりに実機を提供することは現実的ではない。学習者がネットワーク技術を効果的に習得するため、講義資料や演習問題などに加えて仮想ネットワークの構築演習を実現するためのシステムを提案する。講義資料等の提供は汎用のオンライン学習管理システムMoodleを用い、仮想ネットワークの構築と動作確認は独自の判定システムを作成し使用する。この学習システムが多数の学習者の同時アクセスに耐えうるものかを検証するため、同時リクエスト数およびコネクション数を変化させて性能評価実験を行う。経過時間がリクエスト数と比例してることがわかった。また、作成したシステムの問題点や使用感を調査するため、アンケート評価を実施した。作成したシステムがネットワーク学習の支援になっているという意見が多かったが、問題点としてシステムの使用方法がわかりにくいことが挙げられた。
執筆分担について、魚本が 1 節「はじめに」、3.3 節「Moodle と独自プラグ イン」、4 節「性能評価実験」、4.1 節「実験手順」、4.2 節「実験結果」、6 節「ま とめと課題」について担当した。大須賀が 2 節「関連研究」、3.2 節「独自の判 定システム」、5 節「アンケート評価」、5.1 節「アンケート手順」、5.2 節「アン ケート結果」について担当した。中村が 3 節「システム概要」、3.1 節「UIの構成」について担当した。
