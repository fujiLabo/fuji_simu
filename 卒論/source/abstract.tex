\section*{\center 概 要}
\addcontentsline{toc}{section}{概要}

近年、インターネットの普及が進むにつれ、情報技術者にとってネットワーク技術への理解は必要不可欠なものであると同時に、座学などを用いて知識としてネットワーク技術を学習しても、実際のネットワークと学習したネットワーク技術の知識が繋がりづらい分野である。実際にネットワークを構築し、機器情報などを追加することで、正しいネットワークを形成する演習を行うことが実際のネットワークに関する知識を深めるのに効果的だと考えられる。しかし、教育機関や学習者である個人が、ネットワーク構築の演習に必要な機器をすべて揃え、それらを用いてネットワークの構築を行うのは、あまり現実的ではない。解決策として、eラーニングを用いた自己学習が挙げられる。\\
 また、多くの企業や教育機関においてLMSを用いてのeラーニング学習が行われている。しかし、LMSが行うのは学習の管理である。より高度な学習をLMS上で行うには、学習したい内容に合わせた学習支援ツールをLMSに導入しなければならない。\\
 そこで、これらの問題を解決するため、本研究では、LTIに準拠した学習支援ツールとして、ネットワーク自己学習機能を保持したWebアプリケーションの実装を提案する。学習支援ツールは独立したWebアプリケーションとして機能しているので、様々なLMSから呼び出すことが可能である。本研究ではLTIに準拠したLMSとしてCanvasとMoodleをそれぞれ導入し、LTIに準拠した学習支援ツールとしてネットワーク自己学習機能を持ったWebアプリケーションを導入した。
