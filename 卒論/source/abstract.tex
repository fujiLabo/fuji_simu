\section*{\center 概 要}
\addcontentsline{toc}{section}{概要}
従来から系譜図には2D表現が用いられてきた。系譜図は登場人物が増えることで
関係を示す線が交差してしまうという欠点がある。
この問題を3Dで系譜図を表現することによって解決できないか試みた。

本研究では、3D系譜図自動描画システムを作成した。
その上で第三者からの評価を得るために、プラットフォームをWebとして開発を行った。
さらに、作成したシステムを評価する方法を提案した。

その結果、3Dにしたことのみで視認性が上がったとは言い難いが、拡大縮小などの機能を使用することでユーザーにとって
見やすい情報量に適宜変更することが可能になった。

% 本研究ではギリシャ神話の系譜図をデータとして扱う。\ref{tag:gmythology}にて詳細を示すが、ギリシャ神話は複雑な
% 関係から成り立っているため本研究のデータに用いることとし、それらにおいて発生する問題点を解消しつつを3D表現の
% 系譜図を実現化することを目指す。
