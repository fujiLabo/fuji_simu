\section{まとめと課題}
\label{tag:summary}
 本研究では、LTIに準拠することで、複数のLMSで同じように使用することのできる、学習支援ツールとしてのネットワーク自己学習機能を保持したWebアプリケーションを提案した。また、この独立したWebアプリケーションがLTIに準拠していることを示すために、LTIに準拠したLMSであるCanvasとMoodleを異なる仮想マシン上に実装し、これらとは異なる仮想マシン上に実装したネットワークシミュレータを学習支援ツールとして呼び出した。この際、Canvas,Moodleの両者から同じようにネットワークシミュレータとしての機能を使用し、ネットワークシミュレータ内での動作に応じてLMS側に得点を反映できることを確認した。これにより、本研究で実装したネットワークシミュレータはLTIに準拠しており、LTIに準拠したLMSからならどんなLMSからでも呼び出すことが可能である。\\
 本研究で実装したネットワークシミュレータは学習支援ツールとしての使用を前提としていたにもかかわらず、LMS側との連携は得点の反映しか行っておらず、これではLMS側の採点機能しか活用できていない。今後の課題として、ネットワークシミュレータでの問題の作成、作成した問題の共有、公開などの機能の追加が挙げられる。これにより、グループ間で自分が作成した問題を共有したり、他の学習者が作成した問題に取り組んだりと、LMSとしての機能を活用することでネットワークの知識の定着をより強めることができると考えられる。また、本研究で実装されたネットワークシミュレータはネットワーク層のルーティングに関する構築演習しか実装されておらず、今後データリンク層やアプリケーション層などの機能の追加やセキュリティの概念としてファイアウォールの機能の実装が期待される。
