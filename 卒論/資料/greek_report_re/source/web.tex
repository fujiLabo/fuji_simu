% WEBレイアウト
\subsection{Webページデザイン}
\label{tag:web}
Webで公開するにあたりWebページのデザインを提案する。
本研究では、わかりやすくシンプルなデザインのWebページを作成すること、そして訪問者が迷わず操作できることを目標とした。
本研究で作成したWebページは、HTML,CSSおよびJavaScriptで記述している。
単一のページにコンテンツをまとめると訪問者がわかりづらくなってしまうため、情報を分散させるためにWebサイトを
「トップページ」「概要ページ」「描画ページ」「アンケートページ」の4つのページに分けた。
以下で各ページの詳細を説明する。

図\ref{fig:top}は訪問者が最初に閲覧するトップページである。
このページが基本ページになっており、他ページの説明と選択したページに移動できるようになっている。
また、訪問者が移動したいページがわからなくなってしまった場合、このページに戻り、移動先がわかるようになっている。

図\ref{fig:gai}は、概要ページである。
描画システムの概要が書かれている。訪問者が概要、特徴を想像しやすいように、アイコンを使用して簡潔に説明をしている。

図\ref{fig:byou}は、描画ページである。
本研究のメインであるギリシャ神話の系譜図自動描画が見られるページである。

図\ref{fig:ank}は、アンケートページである。
描画システムの訪問者からアンケートを取り、今後の改善に役立てることを目標とする。
また、訪問者がアンケートに答えやすいようにアンケートページは新規タブで開かれるようになっている。

\begin{figure}[htbp]
  \begin{center}
    \includegraphics[clip,width=12.0cm,height=8.0cm]{./img/top.png}
    \caption{トップページ}
    \label{fig:top}
  \end{center}
\end{figure}

\begin{figure}[htbp]
  \begin{center}
    \includegraphics[clip,width=12.0cm,height=8.0cm]{./img/gai.png}
    \caption{概要ページ}
    \label{fig:gai}
  \end{center}
\end{figure}

\begin{figure}[htbp]
  \begin{center}
    \includegraphics[clip,width=12.0cm,height=8.0cm]{./img/byou.png}
    \caption{描画ページ}
    \label{fig:byou}
  \end{center}
\end{figure}

\begin{figure}[htbp]
  \begin{center}
    \includegraphics[clip,width=12.0cm,height=8.0cm]{./img/ank.png}
    \caption{アンケートページ}
    \label{fig:ank}
  \end{center}
\end{figure}