\subsection{データベース}
\label{tag:database}
本節では、ギリシャ神話の系譜図から作成したデータベースと、データベースへのデータ追加や変更を行うデータベース管理プログラムの
機能について説明する。

\subsubsection{データベースの作成}
\label{tag:createDB}
  神々の関係について記入したデータベースの作成を目指した。データベースは千葉\cite{chiba} が作成したA4用紙388枚に
  およぶ紙面上の系譜図をメインとして、一部は『神統記』\cite{theogony}、『ギリシャ神話』\cite{mythology}を参照する。
  データベースは個人ノードに必要な項目を持ち、
  それぞれ「ID」「名前(日本語)」「名前(英語)」「名前(ギリシャ語)」「性別」「親のID」「婚姻相手のID」「参照した元と
  なるデータの場所」「同名で別人の可能性がある場合の識別番号」とする。

  「ID」は他の神を参照する際に使用する。名前が重複するものもいるため、名前による参照は行っていない。名前の項目に日本語、
  英語、ギリシャ語の3言語を用いたのは千葉からの要望のためである。「性別」は男もしくは女を示す値が入力されていない
  場合は不明とする。「親のID」は子供から親を参照するために記入する。なお「親のID」については親が一柱のみの場合と男女の
  両親がいる場合があり、加えて親の組み合わせが数通りある場合もある。「婚姻相手のID」については、複数の夫もしくは妻が
  いる場合はそれらを全て記入し、リストとして管理する。「参照した元となるデータの場所」はデータの修正や再度参照する際に
  紙面から探し出す手間を減らす目的で設ける。「同名で別人の可能性がある場合の識別番号」については、日本語名は同一で
  あっても英語名が異なる場合や、千葉によって同一人物の可能性があるが敢えて別個に記入されている場合、
  名前や性別などのプロフィールが不明な場合、異説がある場合は全て別々にレコード\cite{maria} を設け、それぞれを別人と
  して扱う。

 また、「親のID」を参照して「子のID」を抽出させるスクリプトを作成し、それをもとに「子のID」項目を補完した。

\subsubsection{データ追加・変更}
\label{tag:addDB}
Python3で記述したスクリプトを作成し、それを利用してデータベースへのデータ追加・変更を行っている。
新規データベースを作る場合はCSVファイルを読み込み、データの一括追加を行うことができる。
変更・修正をする場合はコンソールで名前を検索し、該当した名前を一度表示してから変更するかを確認する。
名前検索し、該当しない場合はデータを新規に追加する。
名前検索した際に複数の名前が該当した場合、該当した名前をすべて表示する。
変更したい項目を選択し変更内容を入力する。
新規追加の場合、「名前(日本語)」「性別」「親のID」「子のID」「婚姻相手のID」をそれぞれ入力し追加する。
データの変更を確認せずに行うとデータベースの破壊が起きてしまう可能性があるため、
変更する場合は必ず確認ができるように変更前・変更後をそれぞれ表示している。
さらに新規追加の場合は追加したかを明確にするために確認文を表示する。

データ追加・変更の流れは以下の通りである。
\begin{enumerate}
    \item コンソールで名前検索する。
    \item データベースに存在するか確認する。
    \item 存在するなら変更項目選択、しないならば新規登録する。
    \item 内容を入力する。
    \item 入力内容を確認した後、終了する。
\end{enumerate}