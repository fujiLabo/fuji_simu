% 予稿
\documentclass[twocolumn, a4paper]{jsarticle}
\usepackage[dvipdfmx]{graphicx}
\usepackage[dvipdfmx]{hyperref}
\usepackage{ascmac}
\usepackage{listings}
\usepackage{fancyhdr}
\usepackage{pxjahyper}
\usepackage{geometry}
% 余白設定
% \setlength{\textheight}{\paperheight}
% \setlength{\topmargin}{-3.0truemm}
% \addtolength{\topmargin}{-\headheight}
% \addtolength{\topmargin}{-\headsep}
% \addtolength{\textheight}{-55truemm}
% \setlength{\textwidth}{445pt}
% \setlength{\oddsidemargin}{5truemm}
% \setlength{\hoffset}{2.6mm}
% \setlength{\voffset}{-5.4mm}
% \setlength{\evensidemargin}{-2.5truemm}
% \addtolength{\textwidth}{-15truemm}
% \setlength{\columnsep}{3zw}

% 行間の倍率
\renewcommand{\baselinestretch}{1.0}

\geometry{top=2cm, bottom=2cm, left=2.8cm, right=2.8cm}

% maketitleの編集
\makeatletter
  \def\@maketitle{%
    % \clearpage\null
    \begin{center}%
      \vskip 0.0em
      \let\footnote\thanks
      {\@title \par}%
      \vskip 0.0em
      {\normalsize
        \lineskip 0.0em
        \begin{tabular}[t]{c}%
          \@author
        \end{tabular}\par}%
      \vskip 0.0em
      % {\normalsize \@date}%
    \end{center}%
    \par\vskip 0.0em
    \ifvoid\@abstractbox\else\centerline{\box\@abstractbox}\vskip0.0em\fi
  }
\makeatother

% sectionの余白編集
\makeatletter
  % sectionの下マージンを小さく
  \renewcommand{\section}{%
    \@startsection{section}{1}{\z@}%
    {0.1\Cvs}{0.1\Cvs}%
    {\normalfont\large\headfont\raggedright}}
\makeatother

% 図表の余白編集
\setlength\textfloatsep{1truemm}

% 本文中の図のスペース
\setlength\intextsep{10pt}

% ヘッダ情報
\fancypagestyle{normal}{%
    \lhead{}
    \chead{}
    \rhead{}
    \lfoot{}
    \cfoot{\thepage}
    \rfoot{}
    \renewcommand{\headrulewidth}{0.0pt}
}

% タイトル情報
\title{\fontsize{10pt}{0pt}\selectfont 論文番号 fm2018-03\\ \fontsize{14pt}{0pt}\selectfont LTIに準拠したネットワーク\\自己学習機能の提案と実装}
\author{\fontsize{11pt}{0pt}\selectfont 15RD093 菅原 良太, 15RD150 沼田 悠貴\\ \\指導: 藤本 衡 准教授}
\date{}
\hypersetup{
  pdfauthor={15RD093 菅原 良太, unko15RD150 沼田 悠貴},
  pdftitle={LTIに準拠したネットワーク学習支援ソフトの開発},
  pdfkeywords={},
  pdfsubject={},
  pdfcreator={},
  pdflang={Japanese},
  pdfborder={0 0 0},
  colorlinks=false,
}

\begin{document}
\twocolumn[
\pagestyle{normal}
% タイトル
\maketitle
]
\thispagestyle{normal}

\fontsize{9pt}{0pt}\selectfont
% 研究目的
\section{はじめに}
\label{tag:first}
 近年、多くの企業や教育機関においてLMS(Learning Management System)を用いてeラーニングが行われている。しかし、LMSが行うのは学習の管理であり、教材や資料の配布、簡単なテストや課題の実施に、それに対する評価を行うのが主な仕事である。より高度な学習をLMS上で行うには、学習したい内容に合わせた学習支援ツールをLMSに導入しなければならない。この学習支援ツールは特定のLMS上での動作を想定して設計されており、同一のLMS上でしか動作できず、また、LMS上で動作するためには逐一学習支援ツールをインストールし、プラグインとして動作するための細かな設定を行わなければならない。\\
 また、インターネットの普及が進むにつれ、情報技術者にとってネットワーク技術への理解は必要不可欠なものであると同時に、座学などを用いて知識としてネットワーク技術を学習しても、実際のネットワークと学習したネットワーク技術の知識が繋がりづらい分野である。そこで、実際にネットワークを構築し、機器情報などを追加することで、自らの手で正しいネットワークを形成する演習を行うことが実際のネットワークとそれに付随する知識を深めるのに効果的だと考えられる。しかし、教育機関や学習者である個人が、ネットワーク構築の演習に必要な機器をすべて揃え、それらを用いてネットワークの構築を行うのは、あまり現実的ではない。\\
 これらの問題を解決するため、本研究では、LTI(Learning Tools Interoperability)に準拠した学習支援ツールとして、ネットワーク自己学習機能を保持したWebアプリケーションの実装を提案する。LTIに準拠した学習支援ツールであれば、LTIに準拠したLMSから呼び出すことができる。これによって逐一インストールする必要がなく、学習支援ツールは独立したWebアプリケーションとして機能しているのでLTIに準拠したLMSならば、様々なLMSから呼び出すことが可能である。本研究では異なる仮想マシン上にLTIに準拠したLMSとしてCanvasとMoodleをそれぞれ導入し、LTIに準拠した学習支援ツールとしてネットワーク自己学習機能を持ったネットワークシミュレータを導入した。異なるLMSであるCanvasとMoodleからLTIに準拠した学習支援ツールであるネットワークシミュレータの機能を同じように使用し、ネットワークシミュレータでの動作に応じた得点をLMS側に反映することでLTIに準拠したネットワーク自己学習機能の実装とした。

% whitebase
\section{LTI}\label{tag:LTI}
LTI(Learning Tools IterOperability)とは、異なるプラットフォーム間における学習支援ツールの相互運用を可能にするための規格[5]であり、ツール間の通信プロトコルはHTTP上でのメッセージ交換として実装されている。\\

LTIでは、ユーザ情報や課題の進行状況を直接管理するLMSをツールコンシューマと呼ぶ。
標準機能としてLTIに対応しているLMSとして、Canvas, Moodle, Sakai, Blackboardなどがある。
一方、個別の具体的な問題や教材を提供するプラットフォームをツールプロバイダと呼ぶ。LTIに準拠したツールプロバイダを実装すれば、LTIに対応した複数のLMSからツールを実行することが可能となる。
これはツールをプラグインとして実装するのに比べてソフトウェア開発効率の面で極めて有利である。

ユーザとツールコンシューマ間では、ユーザーIDとPWを用いて認証を行う。しかし、ユーザーがツールコンシューマでツールプロバイダを使用する際は、IDとPWを再度入力せずに使用することができる。\\
これがLTIの利点であり、この認証を省くためにOAuthと呼ばれるプロトコルが使われている。
\subsection{OAuth}
OAuth(オーオース)とは、SNSやWebサービス間で「アクセス権限の認可」を行うためのプロトコルである。また、OAuthには1.0と2.0が存在しているが、本研究ではLTI1.0の実装にあたりOAuth1.0を使用している。
%\subsection{LTI1.0におけるOAuth1.0実装手順}

OAuth1.0実装にあたり、第三者による不正なログインを防ぐためのOAuth signature(署名)及びkey(暗号)の作成をする関数をRubyで自作した。\\
署名及び暗号の作成手順を以下に示す。\\
1.「キー」を作成\\
2.「文字列」の作成\\
3.「キー」と「文字列」用いて署名を作成\\
%\subsubsection{キーの作成}
キーの作成\\
「oauth\_consumer\_secret」、「oauth\_token\_secret」をURLエンコードし、&で繋げれば完成。\\
本研究では「oauth\_consumer\_secret」を設定し、「oauth\_token\_secret」は存在させなかった。また、各々をURLエンコードし、「oauth token secret」を空白とし、\&のみを繋げてKeyを作成した。\\
%\subsubsection{文字列の作成}
文字列の作成\\
1.パラメータをアルファベット順に並べ、キー=値...の形で並べた上で,URLエンコードする。\\
2.リクエストメソッド、リクエストURLをURLエンコードする。\\
3.リクエストメソッド、リクエストURL、パラメータの順で\&で繋げることで文字列を作成した。\\
%\subsubsection{署名の作成}
署名の作成\\
1.LTI1.0ではHMAC-SHA1方式を採用しているため、作成した「キー」と「署名」を用いてHMAC-SHA1方式でハッシュ値を生成する。この時バイナリデータでハッシュ値を生成する必要がある。\\
2.生成したハッシュ値を、base64エンコードすることで署名を作成。\\
\subsection{成績反映}
ツール・コンシューマから成績を返すパラメータ「lis\_outcome\_service\_url」を設定し、特定のユーザーを一意的に示す、「SourcedId」をパラメータ「lis\_result\_sourcedid」から取得し、XML内の「SourcedId」を書き変え、ツール・プロバイダでまとめた点数をXML内の「textString」に加えた上で送信。\\

% システム概要
\section{システム概要}
\subsection{システム}
\label{tag:function}
本研究では、プラグインとしてLMS上に新しい機能を提供するのではなく、LTIに準拠したWebアプリケーションを用いて、異なるLMSで同様の機能が提供でき、Webアプリケーション側での操作に対しLMS側に特定の点数を返すことを目的とした。そこで、異なる仮想マシン上にそれぞれLMSであるCanvas、Moodleと、独立したWebアプリケーションとしてネットワーク自己学習機能を導入した。また、ネットワーク自己学習機能はRuby on Railsを用いて実装した。これらは図\ref{fig:virtualMachine}で表しているようにネットワーク自己学習機能は実際には独立したWebアプリケーションであるが、あたかもLMS側にプラグインとして導入されているように機能を提供する。

\begin{figure}[htbp]
  \begin{center}
    \includegraphics[scale=0.3]{img/virtualMachine.png}
    \caption{仮想マシンの構成}
    \label{fig:virtualMachine}
  \end{center}
\end{figure}



また、Webアプリケーション側はLMSに呼び出された際、独立したWebアプリケーションとしてネットワーク自己学習機能を提供する。この機能にはネットワークを自由に構成し、機器情報を設定することのできる自由描画モードと、予め問題として構成されたネットワークに正しい機器情報を追加することで正しいネットワークの作成を目指す問題演習モードが有る。問題演習モードでの正誤によって得られた得点をLMS側に返すことでLMSでの学習者の評価を行う。\\
魚本ら[1]の制作したネットワーク自己学習機能はMoodleの独自プラグインとしてネットワーク自己学習機能を実装している。
クライアントサイドである独自プラグインとしてのシミュレータ部分はHTMLとJavaScriptで、Moodleのプラグインとしての設定の部分はPHPで、シミュレータで作成されたネットワークの構成の正誤の判定プログラムはRubyでそれぞれ記述されている。これは、様々なシステムを使用しているため、複数のシステム間でデータのなどの連携を行わなければならず、安定性にかけていた。\\
 そこで、本研究ではすべてのシステムをRuby on Railsの中で実装した。MVCアーキテクチャに基づいて設計することにより、魚本、大須賀、中村(2018)らの図\ref{fig:beforeCons}で構成されたシステムをすべてRuby on Rails内で実現した。これにより、複数のシステム間でのデータの送受信などを行う必要性がなくなり、システムとしての安定性を実現した。

% アンケート
\subsection{Ruby on Rails}
\label{tag:rails}
本研究で提案したネットワークシミュレータは、Ruby on Railsを用いて実装されている。Ruby on Railsとは、Rubyで構築された、Webアプリケーションを開発するためのフレームワークである。

% まとめ課題
\subsection{UIについて}
\label{tag:ui}
UIの基本的な部分は、魚本、大須賀、中村(2018)らの制作したネットワーク自己学習機能を採用した。これの概要を図\ref{fig:simu}に示す。

\begin{figure}[htbp]
  \begin{center}
    \includegraphics[clip,width=12.0cm,height=8.0cm]{img/simu2.png}
    \caption{ネットワーク自己学習機能 UI}
    \label{fig:simu}
  \end{center}
\end{figure}

図\ref{fig:simu}のネットワーク自己学習機能は、実際にネットワークに関する学習を終えた学生に対しアンケートを行い、9割以上の学生がデザインについて見やすいと答えていた。これにより図\ref{fig:simu}のネットワーク自己学習機能のUIは変更する必要性がないと判断した。\\
 図\ref{fig:simu}は5つの部分に分けられており、機器部、ナビゲーション部、ネットワーク部、機器詳細部、コンソール部となっている。また、図\ref{fig:simu}では自由描画モードと問題演習モードの2つのモードが用意されている。自由描画モードの際、ナビゲーション部ではそれぞれのアイコンをクリックすることでモードの変更、構築したネットワークの正誤の判定、それぞれの機器の詳細情報の確認、すべての要素の削除を行うことができる。\\
 問題演習モードの際は、これに加え練習問題一覧の表示、現在の状況のセーブ、セーブした状態のロード、問題演習モードの終了を行うことができる。\\
 機器部では自由描画モードの際に、PCやルータのネットワーク部へのドロップ、LANモードのON,OFFの切り替えを行うことができる。LANモードがONの場合ネットワーク部へドロップした機器をLANでつなぐことができる。LANモードがOFFの場合、ネットワーク部では機器部からドロップした機器を自由に移動することができる。\\
ネットワーク部では構築されているネットワークのそれぞれの機器に必要な情報を追加する事ができる。これによって正しいネットワークを構築していくことが本ネットワーク自己学習機能の目的である。\\
 機器詳細部はネットワーク部に追加されたそれぞれの機器の情報を確認する部分である。トポロジーの詳細部分の横のプラスボタンを押すことで機器の詳細な情報の表示、マイナスボタンでその非表示を設定することができる。\\
 コンソール部は不可能な操作やエラーなどの不具合が起こった場合などにそれぞれの理由や結果などをコンソールとして入力される部分である。また、ナビゲーション部ののボタンを用いて、ネットワークの正誤を判定する際に、ネットワーク部に構築されたネットワークが正しいかかどうか、間違っている場合構築したネットワークのどこの部分が間違っているかを表示する。\\
 これらの機能により、学習者はPCを複数用意し、実際にネットワークを構築することなくネットワーク自己学習機能上で擬似的にネットワークの構築を行うことができる。これにより、知識として学習しただけでは分かりづらいネットワークの分野を、視覚的に構築することで実際のネットワークの構成などを理解する助けとなる。


\section{実装試験}
\label{tag:experiment}
本研究ではLTIを用いることで実際に複数のLMSから、独立したWebアプリケーションであるネットワーク自己学習機能を同じように学習支援ツールとして呼び出すことができるのかを確認するために、LTIに準拠したLMSであるMoodle、Canvasを用いての実装実験を行った。\\

\subsection{LTI使用方法}
LMS上でTool Provider(ツール・プロバイダ)を使用するには、各LMS上で外部ツールの設定を変更する必要がある。例として、moodleでの使用方法を説明する。\\
moodleでは外部ツール設定より図\ref{fig:moodle config}参照、ツール名、ツールURL、コンシューマキー、秘密鍵の設定をする必要がある。これらの設定を得て、moodleからTool Provider(ツール・プロパイダ)を利用することが可能となる。\\
\begin{figure}[htbp]
  \begin{center}
    \includegraphics[scale=0.3]{img/moodleSet.png}
    \caption{moodle 外部ツール設定画面}
    \label{fig:moodle config}
  \end{center}
\end{figure}

\subsection{成績反映}
Moodleにおいて、実際に成績反映できるかどうかの実験を行った結果を図\ref{fig:moodle score}に示す。
\begin{figure}[htbp]
  \begin{center}
    \includegraphics[scale=0.2]{img/score.png}
    \caption{moodle 成績反映}
    \label{fig:moodle score}
  \end{center}
\end{figure}


\section{まとめと課題}
\label{tag:summary}
本研究では、LTIに準拠することで、複数のLMSで同じように使用することのできる、学習支援ツールとしてのネットワーク自己学習機能を保持したWebアプリケーションを提案した。また、この独立したWebアプリケーションがLTIに準拠していることを示すために、LTIに準拠したLMSであるCanvasとMoodleを異なる仮想マシン上に実装し、これらとは異なる仮想マシン上に実装したネットワークシミュレータを学習支援ツールとして呼び出した。この際、Canvas,Moodleの両者から同じようにネットワークシミュレータとしての機能を使用し、ネットワークシミュレータ内での動作に応じてLMS側に得点を反映できることを確認した。これにより、本研究で実装したネットワークシミュレータはLTIに準拠しており、LTIに準拠したLMSからならどんなLMSからでも呼び出すことが可能である。\\
 本研究で実装したネットワークシミュレータは学習支援ツールとしての使用を前提としていたにもかかわらず、LMS側との連携は得点の反映のみで止まってしまった。今後の課題として、問題の作成、共有、公開機能なのど追加が課題である。
%今後の課題として、ネットワークシミュレータでの問題の作成、作成した問題の共有、公開などの機能の追加が挙げられる。これにより、グループ間で自分が作成した問題を共有したり、他の学習者が作成した問題に取り組んだりと、LMSとしての機能を活用することでネットワークの知識の定着をより強めることができると考えられる。また、本研究で実装されたネットワークシミュレータはネットワーク層のルーティングに関する構築演習しか実装されておらず、今後データリンク層やアプリケーション層などの機能の追加やセキュリティの概念としてファイアウォールの機能の実装が期待される。


% 参考文献
\bibliographystyle{junsrt}
\begin{thebibliography}{9999999}
  \bibitem{sendai} 魚本裕太,大須賀旭,中村優, "応答性を向上した
IP ネットワーク個人学習システム", 2018.
  \bibitem{kitazawa} 北澤友基, 井口信和, “クラウド環境を利用した IP ネットワーク構築演習支援システムの開発”, 情報処理学会第 74 回全国大会公演論文集, pp.891-892
  \bibitem{moodle} ”Moodle - Open-source learning platform”, \texttt{<https://moodle.org/>},参照2018-12-22
  \bibitem{canvas} “Canvas”, \texttt{<https://www.canvaslms.com/>},参照2018-12-22
%  \bibitem{key2} 村上幸生, "Basic LTI に準拠した 学習支援ツールの開発とその評価", 2012
  \bibitem{lti} 「IMSGlovalHP、
  \texttt{<https://www.imsglobal.org/specs/ltiomv1p0/specification>}、参照2018-12-22
%  \bibitem{ltiw}ウィキペディア OAuth、\texttt{<https://ja.wikipedia.org/wiki/OAuth>}、参照2018-12-22
\end{thebibliography}

% \nocite{*}
\bibliography{../biblio}

\end{document}
