\section{はじめに}
\label{tag:first}
 近年、インターネットの普及が進むにつれ、情報技術者にとってネットワーク技術への理解は必要不可欠なものであると同時に、座学などを用いて知識としてネットワーク技術を学習しても、実際のネットワークと学習したネットワーク技術の知識が繋がりづらい分野である。実際にネットワークを構築し、機器情報などを追加することで、自らの手で正しいネットワークを形成する演習を行うことが実際のネットワークとそれに付随する知識を深めるのに効果的だと考えられる。しかし、教育機関や学習者である個人が、ネットワーク構築の演習に必要な機器をすべて揃え、それらを用いてネットワークの構築を行うのは、あまり現実的ではない。解決策として、eラーニングを用いた自己学習が挙げられる。プログラミング学習において、eラーニングを用いた自己学習を行うことができるWebサイトが近年普及しつつある。情報技術者にとってネットワーク技術への理解がプログラミングの知識同様、必要不可欠なものになっている現在、ネットワーク技術においてもeラーニングを用いた自己学習の機会を提供することが求められている。\\
 また、多くの企業や教育機関においてLMS(Learning Management System)を用いてのeラーニング学習が行われている。しかし、LMSが行うのは学習の管理であり、教材や資料の配布、簡単なテストや課題の実施、それに対する評価を行うのが主な機能である。より高度な学習をLMS上で行うには、学習したい内容に合わせた学習支援ツールをLMSに導入しなければならない。\\
 ネットワーク自己学習機能の先行研究として、魚本ら[1]は、特定のLMSのプラグインとしてネットワーク自己学習機能を導入した。これは、特定のLMS上での動作を想定して設計されており、同一のLMS上でしか動作できず、また、導入したLMSに強く依存しているため、LMS側に変更があった場合、それに合わせてプラグイン側も変更しなければならず、プラグインとして動作するための細かな設定を行わなければならない。\\
 北澤ら[2]は、仮想LinuxであるUser Mode Linuxを利用してサーバ上に複数の仮想マシンを生成し、それらを仮想ネットワーク機器として扱うことでネットワークシミュレートを行うシステムを開発した。このシステムはWebアプリケーションとして動作し、学習者はアプリケーションを操作することによって仮想ネットワークを構築する。これは、独立したネットワーク自己学習機能として動作しているため、LMSとの連携を行うことができず、採点機能などを利用しようとした場合、LMS側に手動で行わなければならない。\\
 そこで、これらの問題を解決するため、本研究では、LTI(Learning Tools Interoperability)に準拠した学習支援ツールとして、ネットワーク自己学習機能を保持したWebアプリケーションの実装を提案する。LTIに準拠した学習支援ツールであれば、LTIに準拠したLMSから呼び出すことができる。これによって逐一インストールする必要がなく、学習支援ツールは独立したWebアプリケーションとして機能しているのでLTIに準拠したLMSならば、様々なLMSから呼び出すことが可能である。本研究では異なる仮想マシン上にLTIに準拠したLMSとしてCanvasとMoodleをそれぞれ導入し、LTIに準拠した学習支援ツールとしてネットワーク自己学習機能を持ったWebアプリケーションを導入した。異なるLMSであるCanvasとMoodleからLTIに準拠した学習支援ツールであるネットワーク自己学習機能を同じように使用し、Webアプリケーション側での動作に応じた得点をLMS側に反映することでLTIに準拠したネットワーク自己学習機能の実装とした。\\
 以下、2節では、本研究で利用したLTIについて説明する。3節では、本研究で提案したネットワーク自己学習システムに付いて説明する。4節では、実際にLTIを用いての実装実験について説明する。\\
 また、本研究において、LMS側とネットワーク自己学習機能においてのLTIに関する部分を菅原が、ネットワーク自己学習機能のシステムに関する部分を沼田が担当した。
