%本論

\documentclass[11pt, a4paper]{jarticle}
\usepackage[dvipdfmx]{graphicx}
\usepackage[dvipdfmx]{hyperref}
\usepackage{ascmac}
\usepackage{listings}
\usepackage{fancyhdr}
\usepackage{pxjahyper}

%余白設定
\setlength{\textwidth}{455pt}
\setlength{\hoffset}{5truemm}
\addtolength{\textwidth}{-25truemm}

%行間の倍率
\renewcommand{\baselinestretch}{1.0}

%maketitleの編集
\makeatletter
  \def\@maketitle{%
    \clearpage\null
    \begin{center}%
      \vskip 10.0em
      \let\footnote\thanks
      {\LARGE \@title \par}%
      \vskip 4.0em
      {\Large
        \lineskip 0.5em
        \begin{tabular}[t]{c}%
          \@author
        \end{tabular}\par}%
      \vskip 3.0em
      {\Large \@date}%
    \end{center}%
    \par\vskip 0.5em
    \ifvoid\@abstractbox\else\centerline{\box\@abstractbox}\vskip1.5em\fi
  }
\makeatother

% ヘッダ情報
\fancypagestyle{normal}{%
    \lhead{}
    \chead{\large 2018年度 情報システムデザイン学系卒業論文}
    \rhead{}
    \lfoot{}
    \cfoot{\thepage}
    \rfoot{}
    \renewcommand{\headrulewidth}{0.5pt}
}

% タイトル情報
\title{\LARGE 論文番号 fm2016-01(仮)\\ \Huge LTIに準拠したネットワーク\\学習支援ソフトの開発}
\author{15RD093 菅原 良太, 15RD150 沼田 悠貴\\ \\指導: 藤本 衡 准教授}
\date{提出日: 2018年12月25日}
\hypersetup{
  pdfauthor={15RD093 菅原 良太, 15RD150 沼田 悠貴},
  pdftitle={LTIに準拠したネットワーク学習支援ソフトの開発},
  pdfkeywords={},
  pdfsubject={},
  pdfcreator={},
  pdflang={Japanese},
  pdfborder={0 0 0},
  colorlinks=false,
}




\begin{document}
\pagestyle{normal}
%タイトル
\maketitle
\thispagestyle{normal}
\clearpage


%概要
\fontsize{11pt}{28pt}\selectfont
\section*{\center 概 要}
\addcontentsline{toc}{section}{概要}

インターネットの普及に伴い、情報技術者にとって TCP/IP を中心とした ネットワーク技術の理解は必要不可欠である。ネットワークの構築演習として 実機を使用した演習があるが、学習者一人ひとりに実機を提供することは現実 的ではない。学習者がネットワーク技術を効果的に習得するため、講義資料や 演習問題などに加えて仮想ネットワークの構築演習を実現するためのシステム を提案する。講義資料等の提供は汎用のオンライン学習管理システム Moodle を用い、仮想ネットワークの構築と動作確認は独自の判定システムを作成し使 用する。この学習システムが多数の学習者の同時アクセスに耐えうるものかを 検証するため、同時リクエスト数およびコネクション数を変化させて性能評価 実験を行う。経過時間がリクエスト数と比例してることがわかった。また、作 成したシステムの問題点や使用感を調査するため、アンケート評価を実施した。 作成したシステムがネットワーク学習の支援になっているという意見が多かっ たが、問題点としてシステムの使用方法がわかりにくいことが挙げられた。
執筆分担について、魚本が 1 節「はじめに」、3.3 節「Moodle と独自プラグ イン」、4 節「性能評価実験」、4.1 節「実験手順」、4.2 節「実験結果」、6 節「ま とめと課題」について担当した。大須賀が 2 節「関連研究」、3.2 節「独自の判 定システム」、5 節「アンケート評価」、5.1 節「アンケート手順」、5.2 節「アン ケート結果」について担当した。中村が 3 節「システム概要」、3.1 節「UI の構 成」について担当した。

\clearpage

\tableofcontents
\clearpage

%本文ここから
%研究目的
\section*{謝辞}
本研究での題材となるギリシャ神話の系譜図を提供してくださり、多くの意見をくださった
画家の千葉政助氏、アートフォース株式会社の門山光氏、株式会社画天プロジェクトの井田清氏に、感謝の意を込めて謝辞を送りたいと思います。
また、本研究の御指導や実験への協力をして下さいました藤本准教授とシステム評価研究室の皆様に対し、
ここに心より深く御礼申し上げます

\clearpage


%謝辞
\section*{謝辞}
本研究を引き継ぐ際に様々な情報を教えていただいた魚本悠太氏、大須賀旭氏、中村優氏に感謝したいと思います。また、本研究の御指導や実験への協力をして下さいました藤本准教授とシステム評価研究室の皆様に対し、ここに心より深く御礼申し上げます。

\clearpages

%参考文献
\bibliographystyle{junsrt}
\nocite{*}
\bibliography{../biblio}


\end{document}
