%本論

\documentclass[11pt, a4paper]{jarticle}
\usepackage[dvipdfmx]{graphicx}
\usepackage[dvipdfmx]{hyperref}
\usepackage{ascmac}
\usepackage{listings}
\usepackage{fancyhdr}
\usepackage{pxjahyper}

%余白設定
\setlength{\textwidth}{455pt}
\setlength{\hoffset}{5truemm}
\addtolength{\textwidth}{-25truemm}

%行間の倍率
\renewcommand{\baselinestretch}{1.0}

%maketitleの編集
\makeatletter
  \def\@maketitle{%
    \clearpage\null
    \begin{center}%
      \vskip 10.0em
      \let\footnote\thanks
      {\LARGE \@title \par}%
      \vskip 4.0em
      {\Large
        \lineskip 0.5em
        \begin{tabular}[t]{c}%
          \@author
        \end{tabular}\par}%
      \vskip 3.0em
      {\Large \@date}%
    \end{center}%
    \par\vskip 0.5em
    \ifvoid\@abstractbox\else\centerline{\box\@abstractbox}\vskip1.5em\fi
  }
\makeatother

% ヘッダ情報
\fancypagestyle{normal}{%
    \lhead{}
    \chead{\large 2018年度 情報システムデザイン学系卒業論文}
    \rhead{}
    \lfoot{}
    \cfoot{\thepage}
    \rfoot{}
    \renewcommand{\headrulewidth}{0.5pt}
}

% タイトル情報
\title{\LARGE 論文番号 fm2016-01(仮)\\ \Huge LTIに準拠したネットワーク\\学習支援ソフトの開発}
\author{15RD093 菅原 良太, 15RD150 沼田 悠貴\\ \\指導: 藤本 衡 准教授}
\date{提出日: 2018年12月25日}
\hypersetup{
  pdfauthor={15RD093 菅原 良太, 15RD150 沼田 悠貴},
  pdftitle={LTIに準拠したネットワーク学習支援ソフトの開発},
  pdfkeywords={},
  pdfsubject={},
  pdfcreator={},
  pdflang={Japanese},
  pdfborder={0 0 0},
  colorlinks=false,
}




\begin{document}
\pagestyle{normal}
%タイトル
\maketitle
\thispagestyle{normal}
\clearpage


%概要
\fontsize{11pt}{28pt}\selectfont
\section*{\center 概 要}
\addcontentsline{toc}{section}{概要}

近年、多くの企業や教育機関においてLMS(Learning Management System)を用いてeラーニングが行われている。しかし、LMSが行うのは学習の管理であり、教材や資料の配布、簡単なテストや課題の実施に、それに対する評価を行うのが主な機能である。より高度な学習をLMS上で行うには、学習したい内容に合わせた学習支援ツールをLMSに導入しなければならない。この学習支援ツールは特定のLMS上での動作を想定して設計されており、同一のLMS上でしか動作できず、また、LMS上で動作するためには逐一学習支援ツールをインストールし、プラグインとして動作するための細かな設定を行わなければならない。\\
 また、インターネットの普及が進むにつれ、情報技術者にとってネットワーク技術への理解は必要不可欠なものであると同時に、座学などを用いて知識としてネットワーク技術を学習しても、実際のネットワークと学習したネットワーク技術の知識が繋がりづらい分野である。そこで、実際にネットワークを構築し、機器情報などを追加することで、自らの手で正しいネットワークを形成する演習を行うことが実際のネットワークとそれに付随する知識を深めるのに効果的だと考えられる。しかし、教育機関や学習者である個人が、ネットワーク構築の演習に必要な機器をすべて揃え、それらを用いてネットワークの構築を行うのは、あまり現実的ではない。\\
 これらの問題を解決するため、本研究では、LTI(Learning Tools Interoperability)に準拠した学習支援ツールとして、ネットワーク自己学習機能を保持したWebアプリケーションの実装を提案する。異なる仮想マシン上にLTIに準拠したLMSとしてCanvasとMoodleをそれぞれ導入し、LTIに準拠した学習支援ツールとしてネットワーク自己学習機能を持っWebアプリケーションを導入した。これらを用いて、異なるLMSであるCanvasとMoodleからLTIに準拠した学習支援ツールであるネットワーク自己学習の機能を同じように使用し、Webアプリケーション側での動作に応じた得点を採点機能としてLMS側に反映できることを確認した。

\clearpage

\tableofcontents
\clearpage

%本文ここから
%研究目的
\section*{謝辞}
本研究での題材となるギリシャ神話の系譜図を提供してくださり、多くの意見をくださった
画家の千葉政助氏、アートフォース株式会社の門山光氏、株式会社画天プロジェクトの井田清氏に、感謝の意を込めて謝辞を送りたいと思います。
また、本研究の御指導や実験への協力をして下さいました藤本准教授とシステム評価研究室の皆様に対し、
ここに心より深く御礼申し上げます

\clearpage


%謝辞
\section*{謝辞}
本研究での題材となるギリシャ神話の系譜図を提供してくださり、多くの意見をくださった
画家の千葉政助氏、アートフォース株式会社の門山光氏、株式会社画天プロジェクトの井田清氏に、感謝の意を込めて謝辞を送りたいと思います。
また、本研究の御指導や実験への協力をして下さいました藤本准教授とシステム評価研究室の皆様に対し、
ここに心より深く御礼申し上げます。

\clearpages

%参考文献
\bibliographystyle{junsrt}
\nocite{*}
\bibliography{../biblio}


\end{document}
