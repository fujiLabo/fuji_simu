%本論

\documentclass[11pt, a4paper]{jarticle}
\usepackage[dvipdfmx]{graphicx}
\usepackage[dvipdfmx]{hyperref}
\usepackage{ascmac}
\usepackage{listings}
\usepackage{fancyhdr}
\usepackage{pxjahyper}

%余白設定
\setlength{\textwidth}{455pt}
\setlength{\hoffset}{5truemm}
\addtolength{\textwidth}{-25truemm}

%行間の倍率
\renewcommand{\baselinestretch}{1.0}

%maketitleの編集
\makeatletter
  \def\@maketitle{%
    \clearpage\null
    \begin{center}%
      \vskip 10.0em
      \let\footnote\thanks
      {\LARGE \@title \par}%
      \vskip 4.0em
      {\Large
        \lineskip 0.5em
        \begin{tabular}[t]{c}%
          \@author
        \end{tabular}\par}%
      \vskip 3.0em
      {\Large \@date}%
    \end{center}%
    \par\vskip 0.5em
    \ifvoid\@abstractbox\else\centerline{\box\@abstractbox}\vskip1.5em\fi
  }
\makeatother

% ヘッダ情報
\fancypagestyle{normal}{%
    \lhead{}
    \chead{\large 2018年度 情報システムデザイン学系卒業論文unko}
    \rhead{}
    \lfoot{}
    \cfoot{\thepage}
    \rfoot{}
    \renewcommand{\headrulewidth}{0.5pt}
}

% タイトル情報
\title{\LARGE 論文番号 fm2016-01(仮)\\ \Huge LTIに準拠したネットワーク\\自己学習機能の提案と実装}
\author{15RD093 菅原 良太, 15RD150 沼田 悠貴\\ \\指導: 藤本 衡 准教授}
\date{提出日: 2018年12月25日}
\hypersetup{
  pdfauthor={15RD093 菅原 良太, unko15RD150 沼田 悠貴},
  pdftitle={LTIに準拠したネットワーク学習支援ソフトの開発},
  pdfkeywords={},
  pdfsubject={},
  pdfcreator={},
  pdflang={Japanese},
  pdfborder={0 0 0},
  colorlinks=false,
}




\begin{document}
\pagestyle{normal}
%タイトル
\maketitle
\thispagestyle{normal}
\clearpage


%概要
\fontsize{11pt}{28pt}\selectfont
\section*{\center 概 要}
\addcontentsline{toc}{section}{概要}

インターネットの普及に伴い、情報技術者にとって TCP/IP を中心とした ネットワーク技術の理解は必要不可欠である。ネットワークの構築演習として 実機を使用した演習があるが、学習者一人ひとりに実機を提供することは現実 的ではない。学習者がネットワーク技術を効果的に習得するため、講義資料や 演習問題などに加えて仮想ネットワークの構築演習を実現するためのシステム を提案する。講義資料等の提供は汎用のオンライン学習管理システム Moodle を用い、仮想ネットワークの構築と動作確認は独自の判定システムを作成し使 用する。この学習システムが多数の学習者の同時アクセスに耐えうるものかを 検証するため、同時リクエスト数およびコネクション数を変化させて性能評価 実験を行う。経過時間がリクエスト数と比例してることがわかった。また、作 成したシステムの問題点や使用感を調査するため、アンケート評価を実施した。 作成したシステムがネットワーク学習の支援になっているという意見が多かっ たが、問題点としてシステムの使用方法がわかりにくいことが挙げられた。
執筆分担について、魚本が 1 節「はじめに」、3.3 節「Moodle と独自プラグ イン」、4 節「性能評価実験」、4.1 節「実験手順」、4.2 節「実験結果」、6 節「ま とめと課題」について担当した。大須賀が 2 節「関連研究」、3.2 節「独自の判 定システム」、5 節「アンケート評価」、5.1 節「アンケート手順」、5.2 節「アン ケート結果」について担当した。中村が 3 節「システム概要」、3.1 節「UI の構 成」について担当した。


\clearpage

%目次
\tableofcontents
\clearpage

%本文ここから
%研究目的
\section*{謝辞}
本研究での題材となるギリシャ神話の系譜図を提供してくださり、多くの意見をくださった
画家の千葉政助氏、アートフォース株式会社の門山光氏、株式会社画天プロジェクトの井田清氏に、感謝の意を込めて謝辞を送りたいと思います。
また、本研究の御指導や実験への協力をして下さいました藤本准教授とシステム評価研究室の皆様に対し、
ここに心より深く御礼申し上げます

\clearpage

%関連研究
\input{source/relation}
\clearpage

%LTIについて
\section{LTI}\label{tag:LTI}
LTI(Learning Tools IterOperability)とは、異なるプラットフォーム間における学習支援ツールの相互運用を可能にするための規格[5]であり、ツール間の通信プロトコルはHTTP上でのメッセージ交換として実装されている。\\

LTIでは、ユーザ情報や課題の進行状況を直接管理するLMSをツールコンシューマと呼ぶ。
標準機能としてLTIに対応しているLMSとして、Canvas, Moodle, Sakai, Blackboardなどがある。
一方、個別の具体的な問題や教材を提供するプラットフォームをツールプロバイダと呼ぶ。LTIに準拠したツールプロバイダを実装すれば、LTIに対応した複数のLMSからツールを実行することが可能となる。
これはツールをプラグインとして実装するのに比べてソフトウェア開発効率の面で極めて有利である。

ユーザとツールコンシューマ間では、ユーザーIDとPWを用いて認証を行う。しかし、ユーザーがツールコンシューマでツールプロバイダを使用する際は、IDとPWを再度入力せずに使用することができる。\\
これがLTIの利点であり、この認証を省くためにOAuthと呼ばれるプロトコルが使われている。
\subsection{OAuth}
OAuth(オーオース)とは、SNSやWebサービス間で「アクセス権限の認可」を行うためのプロトコルである。また、OAuthには1.0と2.0が存在しているが、本研究ではLTI1.0の実装にあたりOAuth1.0を使用している。
%\subsection{LTI1.0におけるOAuth1.0実装手順}

OAuth1.0実装にあたり、第三者による不正なログインを防ぐためのOAuth signature(署名)及びkey(暗号)の作成をする関数をRubyで自作した。\\
署名及び暗号の作成手順を以下に示す。\\
1.「キー」を作成\\
2.「文字列」の作成\\
3.「キー」と「文字列」用いて署名を作成\\
%\subsubsection{キーの作成}
キーの作成\\
「oauth\_consumer\_secret」、「oauth\_token\_secret」をURLエンコードし、&で繋げれば完成。\\
本研究では「oauth\_consumer\_secret」を設定し、「oauth\_token\_secret」は存在させなかった。また、各々をURLエンコードし、「oauth token secret」を空白とし、\&のみを繋げてKeyを作成した。\\
%\subsubsection{文字列の作成}
文字列の作成\\
1.パラメータをアルファベット順に並べ、キー=値...の形で並べた上で,URLエンコードする。\\
2.リクエストメソッド、リクエストURLをURLエンコードする。\\
3.リクエストメソッド、リクエストURL、パラメータの順で\&で繋げることで文字列を作成した。\\
%\subsubsection{署名の作成}
署名の作成\\
1.LTI1.0ではHMAC-SHA1方式を採用しているため、作成した「キー」と「署名」を用いてHMAC-SHA1方式でハッシュ値を生成する。この時バイナリデータでハッシュ値を生成する必要がある。\\
2.生成したハッシュ値を、base64エンコードすることで署名を作成。\\
\subsection{成績反映}
ツール・コンシューマから成績を返すパラメータ「lis\_outcome\_service\_url」を設定し、特定のユーザーを一意的に示す、「SourcedId」をパラメータ「lis\_result\_sourcedid」から取得し、XML内の「SourcedId」を書き変え、ツール・プロバイダでまとめた点数をXML内の「textString」に加えた上で送信。\\

\clearpage

%システムについて
\input{source/system}
\clearpage

%まとめと課題
\section{まとめと課題}
\label{tag:summary}
本研究では、LTIに準拠することで、複数のLMSで同じように使用することのできる、学習支援ツールとしてのネットワーク自己学習機能を保持したWebアプリケーションを提案した。また、この独立したWebアプリケーションがLTIに準拠していることを示すために、LTIに準拠したLMSであるCanvasとMoodleを異なる仮想マシン上に実装し、これらとは異なる仮想マシン上に実装したネットワークシミュレータを学習支援ツールとして呼び出した。この際、Canvas,Moodleの両者から同じようにネットワークシミュレータとしての機能を使用し、ネットワークシミュレータ内での動作に応じてLMS側に得点を反映できることを確認した。これにより、本研究で実装したネットワークシミュレータはLTIに準拠しており、LTIに準拠したLMSからならどんなLMSからでも呼び出すことが可能である。\\
 本研究で実装したネットワークシミュレータは学習支援ツールとしての使用を前提としていたにもかかわらず、LMS側との連携は得点の反映のみで止まってしまった。今後の課題として、問題の作成、共有、公開機能なのど追加が課題である。
%今後の課題として、ネットワークシミュレータでの問題の作成、作成した問題の共有、公開などの機能の追加が挙げられる。これにより、グループ間で自分が作成した問題を共有したり、他の学習者が作成した問題に取り組んだりと、LMSとしての機能を活用することでネットワークの知識の定着をより強めることができると考えられる。また、本研究で実装されたネットワークシミュレータはネットワーク層のルーティングに関する構築演習しか実装されておらず、今後データリンク層やアプリケーション層などの機能の追加やセキュリティの概念としてファイアウォールの機能の実装が期待される。

\clearpage


%謝辞
\section*{謝辞}
本研究を引き継ぐ際に様々な情報を教えていただいた魚本悠太氏、大須賀旭氏、中村優氏に感謝したいと思います。また、本研究の御指導や実験への協力をして下さいました藤本准教授とシステム評価研究室の皆様に対し、ここに心より深く御礼申し上げます。

\clearpages

%参考文献
%key部分に文書中に参考文献として引用する際のラベル名を入れる
\begin{thebibliography}{99}
  \bibitem{key1} aaaaa
  \bibitem{key2} aa
\end{thebibliography}


\end{document}
