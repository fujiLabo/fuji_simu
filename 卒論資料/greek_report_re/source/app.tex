\subsection{ノードの座標計算}\label{tag:backend}
本節では、データベースのWHIteBasEモデル化と座標計算を行うスクリプトについて説明する。
座標計算スクリプトはPython3で記述され、系譜図の描画に必要な座標を計算する機能と同時に、
系譜図のデータベースをWHIteBasEモデル化する機能を有する。
クライアントで中心に描画したいノードを指定し、アプリケーションサーバにノード名をHTTP通信でPOSTし、
その情報をもとにアプリケーションサーバが座標を計算し、座標を含む個人ノードのデータをレスポンスとしてクライアントに送信する。

データベースへのデータ登録については\ref{tag:database}節で述べている。
系譜図データベースをWHIteBasEモデル化する機能については\ref{tag:wbModel}節、
データをもとに座標計算する機能については\ref{tag:coord}節で説明する。

\subsubsection{WHIteBasEモデル化}\label{tag:wbModel}
WHIteBasEモデルは\ref{tag:wb}節で述べたように、2D系譜図表示における線の交わりを考慮したデータベース管理手法である。
しかしこの手法は、各ノードをオブジェクトとして考え、不可視結節点のオブジェクトを新たに設けることで、プログラムとの親和性が高まっているため、
2D表示であるかにかかわらず、系譜図を表すデータ構造として適していると期待される。
よって、本研究では座標計算を行うためのデータ構造として、WHIteBasEモデルを利用する。

本システムで扱いやすいように、WHIteBasEオブジェクトと個人ノードオブジェクトを作り、WHIteBasEモデルを実現する。
WHIteBasEオブジェクトは\ref{tag:wb}節をもとに、$S_R$、$S_L$、$D$のほか、WHIteBasEオブジェクトIDを要素として持つ。
個人ノードオブジェクトは\ref{tag:database}節をもとに、個人ノードオブジェクト「ID」「名前(日本語)」「性別」「婚姻相手のID」「親のID」
「子のID」「その他情報」「座標」を要素として持つ。

データベースはMariaDBを用いているので、SQLを記述することでデータベースにアクセスすることが出来る。
データには親子や婚姻関係に関するデータが格納されているので、本スクリプトでそのデータを取得する。
クライアントから中心に描画したいノード(以下中心ノード)の名前を受け取り、そのノードをキューに追加する。
その後次の手順をキューが空になるまで繰り返す。
\begin{enumerate}
    \item キューの先頭を取り出し、個人ノードオブジェクトを作成する。
    \item キューの先頭から取り出したノード(以下注目ノードと呼ぶ)の婚姻関係の数だけ空のWHIteBasEオブジェクトを作成し、婚姻者の個人ノードオブジェクトを作成する。
    \item 注目ノードをそれぞれの$S_R$に追加し、各婚姻者をそれぞれの$S_L$に追加する。
    \item 注目ノードの子ノードをすべてキューに追加する。
    \item 注目ノードと婚姻者の間に出来た子供を$D$に追加する($D$はリスト化する)。
    \item 婚姻関係のない子どもがいる場合は、別途空のWHIteBasEオブジェクトを作成し、$S_R$にpopしたノード、$D$に子ノードを追加する。
\end{enumerate}
キューが空になると、中心ノードと血の繋がった、もしくは直接関係のある子方向のノードがすべてWHIteBasEモデル化し、
これを利用して座標計算を行う。

\subsubsection{座標計算}\label{tag:coord}
Webブラウザでグラフィックスを描画する際、画面上での座標を用いるが、
WebGLで描画する際は、HTML5のcanvas要素を利用して描画される。canvas内は直交座標系
\footnote{互いに直交する座標軸を有する座標系である。3次元空間内においては、x軸, y軸, z軸が互いに直交する。}
を持ち、
three.jsではcanvasの中心が座標$(x, y, z) = (0, 0, 0)$となり、
画面右方向がx軸の正方向、画面上方向がy軸の正方向、画面手前(視点)方向がz軸の正方向である。

一般的な2Dの系譜図では、親子関係が上下、婚姻関係は左右に配置されることが多い。
本システムでも、階層すなわち世代を明確に区別できるように、子供はy軸負方向(画面下方向)に配置し、
婚姻関係は原点$(x, y, z) = (0, 0, 0)$から遠ざかるようx-z平面上に配置する。
ここで、本システムでのノードの配置例を図\ref{fig:node}に示す。
図\ref{fig:node}において、赤線は婚姻関係、青線は親子関係を示す。
点線は円周上に並ぶことを明示するための補助線で、実際に系譜図に描画されるものではない。

婚姻関係や子供が複数存在する場合、ノードを中心にその婚姻関係や子供を円状に配置する。
子供は、婚姻関係がない親のもとに生まれた場合は親から直接真下(y軸負方向)に配置し、
そうでない場合は注目ノードの婚姻者の斜め下(y軸負方向かつ原点から離れる方向)に配置する。

WHIteBasEモデル化によって既に木構造のようなデータ構造になっているので、
子に向かって探索しながら次のような手順で座標を決定する。
\begin{enumerate}
    \item 注目するノードの婚姻者を、WHIteBasEを用いてすべて探索する。
    \item 婚姻者の座標を決定する。注目ノードを中心にしたx-z平面の円周上に等間隔で配置する。
    \item 注目するノードと婚姻者の間に出来た子供を、WHIteBasEを用いてすべて探索する。
    \item 婚姻者の座標をもとにその子供の座標を決定する。婚姻者の下に、かつ婚姻者を中心にしたx-z平面の円周上に等間隔で配置する。
    \item 子供を注目するノードに設定する。
\end{enumerate}
以上の手順を再帰的に繰り返し、すべての座標を計算して個人ノードオブジェクトの要素として追加する。

ここまでで作成された個人ノードオブジェクトのリストをJSON形式に変換し、
URLエンコードを行ってクライアントに送信する。

\begin{figure}[htbp]
    \begin{center}
        \includegraphics*[scale=0.4]{img/familytree02.png}
        \caption{ノード配置例}
        \label{fig:node}
    \end{center}
\end{figure}

\subsection{Webブラウザ上における系譜図描画}\label{tag:client}
本節では、Webブラウザ上で表示される系譜図の描画システムのUIについて述べる。
本システムでは、系譜図描画はJavaScriptとそのライブラリであるthree.js\cite{threejs}によって行われる。
three.jsで、指定した場所にCanvas要素を自動で生成し、そのCanvas要素内に系譜図を描画する。
系譜図の描画例を図\ref{fig:familytree}に示す。

Webブラウザにおいて3次元コンピュータグラフィクスを表示させる際、
OpenGL\cite{opengl}をJavaScript\cite{javascript}で制御することができるWebGL\cite{webgl}を用いる。
しかしWebGLのAPI
\footnote{Application Programming Interfaceの略で、ソフトウェアの機能や管理データなどを他のプログラムから呼び出して利用するための手順やデータ形式などを定めた規約を指す。}
は低レベルであり、そのまま使う際に冗長な準備をしなければならないため、
JavaScriptライブラリなどを通してWebGLを利用することで、高レベルAPIとして扱うことができるようになる。
このようなライブラリの一つとして、three.jsが利用されることが多い。

three.jsには様々なジオメトリを作成する機能がある。
本システムでは、「ノードを表す立体オブジェクト」「それらを結ぶ線」「ノードの名前」で構成した。
ノードを表す立体オブジェクトは、性別で形を変え、男性・女性・不明の3つに分けた。
不明な性別は、男性・女性で表示しきれない性別も含む。
ここで、男性は立方体、女性は球、不明は正四面体で表した。
なお、不明を正四面体で表す理由として、ひと目で明確に男性・女性と区別できるようにするためである。
ノードの色は、対比現象により周囲より彩度を高くすることで注目されやすくなる\cite{chromatics}ことから、
何を中心に描いたのかを明示するために、中心となるノードの彩度を高く設定した。
また、ノードを結ぶ線は赤線を婚姻関係、青線を親子関係とした。
ノードの名前はthree.jsのTextGeometryを利用した。
この際、文字のフォントを別途読み込む必要が有るため、TrueTypeFont
\footnote{アップルコンピュータとマイクロソフトが共同開発した、デジタルフォントの符号化方式の1つである。}
をJSON形式に変換するJavaScript製ツールであるFacetype.js\cite{facetype}を利用して文字フォントデータを扱った。

系譜図描画のWebページを開いた際、JavaScriptによって中心に描きたいノードがアプリケーションサーバに送信され、
そのレスポンスとして\ref{tag:wbModel}節で作成されたJSON形式のデータを受け取り、読み込む。
この座標を利用して、各ノードのジオメトリに座標を設定し、各ノードをCanvas要素に描画する。
また、ノード同士を結ぶ線は、婚姻関係と親子関係をデータから参照し、それぞれの座標を端点とする直線を引く。
さらに、各ノードの座標に名前を表示した。

Webブラウザ上で表示された系譜図はCanvas要素上でマウスによる操作が可能である。
これはthree.jsのTrackballControls.jsによって実現している。
ここでは、右利き用マウスで操作説明をする。
系譜図が表示されている画面で、左クリックを押しながらマウスを動かすと原点を中心に回転する。
右クリックを押しながらマウスをうごかすと系譜図全体がパン(x-y方面の移動)する。
また、マウスホイールを動かすと拡大縮小する。
回転はキーボードの'A'を、パンは'S'を、拡大縮小は'D'を押しながら、
かつ左クリックを押しつつマウスを動かすことでも、同様の操作が出来る。

\begin{figure}[htbp]
    \begin{center}
        \includegraphics*[scale=0.4]{img/familytree01.png}
        \caption{系譜図表示の例}
        \label{fig:familytree}
    \end{center}
\end{figure}