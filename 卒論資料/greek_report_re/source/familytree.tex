\section{系譜図について}
\label{tag:tree}
古来より、代々の系統を表現するために系譜図が利用されている。
特定の人物の出自を説明するために両親、兄弟、その人物の婚姻相手、先祖や子孫などを示す資料として現代でも利用されている。
系譜図は特定の家庭における家督相続のみならず、
植物の品種改良の過程や、分離・合流を繰り返す政党の成り行きを表現することもできる。

典型的な系譜図の表現方法は次の通りである。また、図\ref{fig:familytree}にも示す。
歴史的に用いられてきた媒体の制約上、
系譜図が表現すべき時間推移と系統間の相互関係を平面で表現する必要があった。
そこで2人の人物を横に結ぶことで婚姻関係を、
その線から下に伸ばすことで上が親を、下が子供を示す。
すなわち基本的には下方向に進むにしたがって世代が進んでいく。

\begin{figure}[htbp]
  % 系譜図
    \begin{center}
      \includegraphics*[scale=0.3]{img/familytree-ver2.png}
    \end{center}
    \caption{親子関係を示す一般的な系譜図}
    \label{fig:familytree}
\end{figure}