\section{WHIteBasEについて}
\label{tag:wb}

本研究では、前節で示したような複雑な人物や関係をデータベース化するモデルとして、
WHIteBasEモデル\cite{white}を用いる。

WHIteBasE(Widespread Hands to InTErconnect BASic Elements)とは、
系譜図における関係性を扱うための管理手法として、杉山らが2011年に提案したものである\cite{white}。
2Dの系譜図における関係性を示す線分が交叉することを考慮し、
婚姻関係と子の関係を一つのイベントとして不可視結節点(図\ref{fig:whitebase01})を用いて管理される。
この不可視結節点をWHIteBasEと呼ぶ。
ここで、個人のノードを管理するための不可視結節点を使った結合モデルを図\ref{fig:whitebase02}に示す。

WHIteBasEモデルは、ノードを結合するための3種類の「鍵穴」を持つ。
このうち2種($S_L$,$S_R$)はそれぞれ1つずつ存在し、婚姻関係にある2ノードを結合する役割を持つ。
またこの婚姻関係から生じたN個の子ノードを結合するために、鍵穴$D_k$($k=1,2,...,N$)が用いられる。
実際の表示位置としての左右や、男親・女親の違いは順不同である。
また、$D_k$は下向きに子のノードを結合するための「鍵穴」である。

個人ノード(Individual Node)は2種類の鍵A, $M_j$($j=1,2,...,N$)を持ち、
これらの鍵とWHIteBasEの鍵穴により、不可視結節点に個人を結合することを表す。
A(Ascendant)は鍵穴$D_k$に結合して親への鍵を、M(Married)は$S_L$または$S_R$に結合して婚姻の鍵をそれぞれ表す。
子の個人ノードはそれぞれの鍵Aを用いてWHIteBasEと結合される。
婚姻関係の個人ノードに対しては鍵$M_j$を用いてそれぞれWHIteBasEに結合する。

このモデルを用いた座標計算については、\ref{tag:backend}節で述べる。

\begin{figure}[htbp]
  % 1
  \begin{minipage}{0.5\hsize}
    \begin{center}
      \includegraphics*[scale=0.3]{img/whitebase01.png}
    \end{center}
    \caption{2Dでの系譜図表示スタイル}
    \label{fig:whitebase01}
  \end{minipage}
  % 2
  \begin{minipage}{0.5\hsize}
    \begin{center}      
      \includegraphics*[scale=0.4]{img/whitebase02.png}
    \end{center}
    \caption{WHIteBasEモデル}
    \label{fig:whitebase02}
  \end{minipage}
  \label{fig:whiltebase00}
\end{figure}