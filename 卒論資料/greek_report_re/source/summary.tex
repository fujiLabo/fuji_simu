\section{考察}
本研究ではギリシャ神話の系譜図を描画することを目的としているが、
ギリシャ神話は非常に情報が多く複雑な関係が存在しているので、
従来の系譜図表示では関係がわかりにくくなってしまうという問題点がある。
これは\ref{tag:greek}節で挙げた点を解消することで改善がみられると考えられる。
本研究で挙げた問題点のうち線の交わりが増える問題については、
系譜図を3Dで表示することによって解消され、関係の把握をしやすくなったと考えられる。
\cite{ui}において、視覚に入った4〜5個のものの個数を数えずに把握する時間は0.2秒で、
個数が4〜5個を超えると即座の把握は難しくなると記述されている。このことを踏まえると、3Dにしたことのみで
視認性が上がったとは言い難いが、拡大縮小などの機能を使用することでユーザーにとって見やすい情報量に適宜変更することが
できると考えられる。

またアプリケーションの評価をいくつかの質問項目で行っているが、
系譜図はノードの関係を表すグラフであるため、
ユーザがグラフの探索をする際に正しく探索できているかを調査するほかに、
特定のノードまでたどり着くまでの時間を計測することで、
系譜図の探索をスムーズに行えているかどうか数値での評価ができると考えられる。
探索開始から終了までの時間を計測できる機能を付随させることができれば、
より一層システムの改善を行えると期待される。


\section{まとめと課題}
\label{tag:summary}
本研究では、Webブラウザ上でギリシャ神話を元に系譜図を自動で3D描画するシステムを構築した。
本システムでの3D表示により、2D表示では解決できなかった線の交わりと、
複数の婚姻関係・子関係によって系譜図が左右方向に広がることを解消することができた。
また、Webブラウザ上で描画するので特別なソフトウェアを用意する必要はなく、様々な端末で手軽に系譜図を見ることができる。
さらに、マウスを用いて視点を変更することが容易となり、インタラクティブな操作が可能なため利用者が
系譜図をわかりやすく操作することができると期待される。
そして、Webブラウザで答えられるアンケートを作成し利用者からフィードバックを得ることで、
描画システムの改善を図ることができるだろう。

今後の課題としては、親から子への探索、描画はできているが、子から親への探索、
描画ができていないのでわかりやすくするためには解決する必要がある。
また、3Dのため描画された系譜図をマウスを用いて視点変更していくと、
どちらの方向が親方向なのか、子方向なのかわかりにくくなってしまう。
本研究では、アンケートの実施まで至らなかったので、実際にアンケートを取って利用者の意見を参考にして、
描画部分の改善をしていきたいと考える。
アンケートを実施するにあたって、はじめに指定されている人物を見つけ出すことが現状では困難なので、
検索機能を実装し特定の人物を発見しやすくする必要がある。

将来的には、ギリシャ神話の系譜図をより利用者にわかりやすくするために
VRや拡張現実(AR)を利用した系図描画システムを構築することも考えられる。

本研究ではギリシャ神話を対象とした系譜図自動描画を行なったが、
ギリシャ神話以外を対象として関係性の描画をすることもデータベースに情報を持たせることができれば可能であると考える。


% 「UIデザインの心理学」を参考文献として
% 視覚に入った4〜5個のものの個数を数えずに把握する時間は0.2秒で、
% 個数が4〜5個を超えると即座の把握は難しくなる。
% 以上のことから3Dにしたから即座に見やすくなったわけではないが、
% 拡大縮小をすることで自分にとって見やすい情報量に変更することができる。