\section{ギリシャ神話について}
\label{tag:greek}
本研究では3D表現する系譜図の例として、ギリシャ神話に登場する神、人物その他の存在に関する婚姻・親子関係を扱う。
ギリシャ神話には数多くの神々や英雄、怪物が登場する。その神々は人智を超えた現象を引き起こしたり、人間のように
恋愛や争いを繰り広げたりしていたことが伝えられている。数多くの登場人物がいるということは、すなわち数多くの関係が
あったと言える。ギリシャ神話の中でも特に有名なのは全知全能の神ゼウスであるが、絶対的に強力な力を持つ神々の王で
あることに加えて、非常に浮気者であったことも有名である一因となっているだろう。その結果ゼウスは非常に多くの子孫を
残している。

これらの関係を従来の系譜図に書き起こすと膨大な情報量となってしまい、その上同じ名前の神が複数箇所に
登場するようになってしまう。このため、ある一柱の神と血縁的または配偶的な関係を持つ神々を全て探し出すのは非常に
困難である。神々における系譜図を作成するにあたり留意すべき課題を以下に挙げる。
\begin{description}
  \item[複数の婚姻関係と近親婚]\mbox{}\\
    上で述べたように一柱の男神に対して妻となる女神は一柱とは限らない。
    また逆に一柱の女神に対して多数の男神と婚姻関係が存在する例もある。
    また配偶者が近い血縁者であることは珍しくなく、同じ両親から生まれた兄妹で婚姻関係を持っていたり母と息子で持っている例もある。
    ただしこれらの課題はギリシャ神話特有というわけではなく、近親婚は近代以前ではしばしば見られる事例であり、
    複数の婚姻関係は現在でも一般的に起こることである。
  \item[数世代にわたって同じ神が登場する]\mbox{}\\
    一部のエピソードにおいて戦死する例もあるが、基本的に神々は不老不死である。人間における系譜図は下方向に進むに
    したがって世代が進んでいくが、神々における系譜図はそうであるとは限らない。数世代にわたって同じ神が登場することが
    あり、それに伴って関係がより複雑化している。
  \item[男女のペアから誕生したとは限らない]\mbox{}\\
    特定の神の身体の一部分から誕生したものや、女神が自らの力のみで子を産んだものもある。例えば
    ウラーノスの切り落とされた男性器にまとわりついた泡から誕生した神もいる\cite{mythology}。
  \item[後に名前が変わる]\mbox{}\\
    生前と死後では違う名前となる例がある。例えば人間であったセメレーはゼウスの雷光に耐えられず死んでしまうが、その時に
    身籠っていたゼウスとの息子であるディオニューソスが死後の世界からセメレーを連れ戻した後セメレーは女神となり、
    その後天に昇りテュオーネーと名を改めた。
  \item[親が1組とは限らない]\mbox{}\\
    様々な異説により、親が複数パターンにわたるものもある。
  \item[性別がないものもある]\mbox{}\\
    本研究に使用したデータの中には性別のない生命体や無機物、概念も含まれている。例えば恒星や遊星といったものや、
    ガイアが生み出した山などがこれにあたる。
\end{description}
データ表現を決定するにあたっては、以上の点を考慮する必要がある。\\
